
\subsubsection{STRUCT} \index{STRUCT} \index{ENDSTRUCT}
\begin{verbatim}
STRUCT <prototype name> [,<variable name>]
...
ENDSTRUCT
\end{verbatim}

\begin{xen}
As Z80 processor is able to manage structured data thanks to its IX and IY registers, RASM introduces STRUCT directive, wich is used for defining a structure, in a similar way to C syntax
\end{xen}

\begin{code}
; structure st1 created with two fields ch1 and ch2.
struct st1
 ch1 defw 0
 ch2 defb 0
endstruct

; Nested structures:
; metast1 is created with 2 sub-structures st1 called pr1 et pr2
struct metast1
 struct st1 pr1
 struct st1 pr2
endstruct
\end{code}



\begin{xfr}
\paragraph{Instanciation}

Quand \texttt{\{STRUCT\}} est utilisé avec 2 paramètres, RASM va créer une structure en mémoire, basée sur le prototype.
Dans l'exemple ci dessous, une structure de type 'metast1' sera instanciée,avec pour nom 'mymeta':
\end{xfr}

\begin{xen}
When \texttt{\{STRUCT\}} directive is used with 2 parameters, RASM will create a structure in memory, based on the prototype.
In the example below, it will instantiate a metast1 structure type, called mymeta.
\end{xen}

\begin{code}
struct metast1 mymeta
\end{code}

\begin{xfr}
Exemple de récuperation de l'adresse absolue d'un membre en utilisant le nom d'une structure déclarée:
\end{xfr}

\begin{xen}
Example of retrieving fields absolute address using the structure previously declared:
\end{xen}

\begin{code}
LD HL,mymeta.pr2.ch1
LD A,(HL)
\end{code}

\begin{xfr}
Exemple d'accès d'un membre avec un offset, en utilisant le nom du prototype:
\end{xfr}
\begin{xen}
Example of accessing a field with an offset, by using the prototype name:
\end{xen}

\begin{code}
LD A,(IX+metast1.pr2.ch1)
\end{code}

\subsubsection{SIZEOF}\index{SIZEOF}

\begin{xfr}
  Le préfixe \texttt{\{SIZEOF\}} devant un nom de structure, de sous-structure ou meme devant le champ d'une structure, permet de récupérer sa taille.
%Il est recommandé, pour recuperer la taille d'une structure, d'utiliser le prefixe \{SIZEOF\}.
\end{xfr}

\begin{xen}
%Using \texttt{\{SIZEOF\}} prefix before a structure name get the size of it.
Recommended usage to get the size of a structure is to use \{SIZEOF\} prefix.
It alsoworks for a substructure or a field.
\end{xen}

\begin{code}
  LD A,\{SIZEOF\}metast1              ; LD A, 6
  LD A,\{SIZEOF\}metast1.pr2          ; LD A, 3
  LD A,\{SIZEOF\}metast1.pr2.ch1      ; LD A, 2
\end{code}

\begin{xen}
Like Vasm, you also can get the structure size using its prototype name but it is not recommended.
\end{xen}


\begin{xfr}
\subsubsection{Tableau de STRUCT}
Il est possible d'instancier plusieurs fois la meme structure, comme s'il s'agissait d'un tableau de structures, de la facon suivant:
\end{xfr}

\begin{xen}
\subsubsection{STRUCT Array}
It's possible to instanciate an array of structs, using this syntax:
\end{xen}

\begin{code}
    struct mystruct my\_instances,10
\end{code}

\begin{xfr}
Cette facon de faire est différente d'un \texttt{ds 10*SIZEOF(mystruct)}, car les données sont initialisées avec les valeurs par défaut définies dans la définition de la structure (a la différence d'un remplissage avec le meme octet)
De plus il est possible d'acceder aux structures via un index, en ajoutant l'index a la fin du nom de l'instance (la syntaxe de cette feature sera peut etre modifiée a l'avenir). Par exemple:
\end{xfr}

\begin{xen}
Compared to \texttt{ds 10*SIZEOF(mystruct)}, data is initialized with default values as defined in structure declaration, and not filled with a zero value.
Also it is possible to access to a specific instance, using an index, like a regular array.
\end{xen}

\begin{code}
  LD HL,myinstances5
\end{code}
