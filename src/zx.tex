\section{\xlang{Directives et options spécifique au ZX}{ZX Specific features}}\index{ZX}\label{PZX}

\begin{xfr}
RASM intègre aussi quelques options et directives qui ne concernent que l'architecture spécifique du ZX.
\end{xfr}

\begin{xen}
RASM also features a few options and directives specific for ZX architecture.
\end{xen}


\subsection{\xlang{Directive HOBETA}{HOBETA Directive}}\index{HOBETA}\label{HOBETA}

\begin{verbatim}
SAVE 'filename', start, size, HOBETA
\end{verbatim}

\begin{xfr}
Cette directive est un parametre supplémentaire pour la directide \texttt{SAVE}, de facon à générer un fichier avec une entete HOBETA, en ajoutant l'entete adaptée.
Hobeta est un format pour stocker des fichiers TR-DOS, utilisé principalement par les anciens émulateurs ZX Spectrum. Le fichier Hobeta est essentiellement une copie d'octet du fichier TR-DOS avec un en-tête de 17 octets:

%Le format a maintenant reçu une nouvelle vie dans l'ordinateur ZX Evolution. 
%La particularité du format est qu'un seul fichier TR-DOS est stocké dans un fichier Hobeta . 
%L'extension de fichier est obtenue en ajoutant le caractère "$" devant l'extension du fichier TR-DOS d'origine .
%Le format est apparu avec le copieur Hobeta du même nom (hobeta.exe 19.11.1990 : "HoBeta V2.0, Copyright (C) 1990, InterCompex, Soviet-Swiss JV" - voir Le Hobbit ).
%Description des formats

\medskip

\begin{tabular}{l|l|l}
\texttt{+0}  & 8 octets & nom de fichier TR-DOS \\
\texttt{+8 } & 1 octet  & type de fichier (extension) TR-DOS \\
\texttt{+9 } & 2 octets & paramètre de fichier START \\
\texttt{+11} & 2 octets & paramètre LENGTH du fichier (longueur en octets) \\
\texttt{+13} & 2 octets & taille du fichier en secteurs \\
\texttt{+15} & 2 octets & somme de contrôle des 15 octets précédents (pas le fichier lui-même!) \\
\end{tabular}

%Données:
%Copie d'octet de tous les secteurs du fichier
%L'octet de poids fort de la taille du fichier en secteurs doit toujours être 0.


%Algorithme de calcul de la somme de contrôle:
%\[ \sum_{i=0}^{i<=14}{header[i]*257+i} \]

%Somme de contrôle = 0 ;
%pour (i = 0; i <= 14; CheckSum = CheckSum + (en-tête [i] * 257) + i, i ++);
%Les fichiers binaires (avec l'extension. $ C) dans ce format peuvent être exécutés directement à partir de périphériques FAT (tels qu'un disque dur ou une carte SD) intégré au shell Evo Reset Service.

\end{xfr}

\begin{xen}
Use this directive for generating a file in \texttt{HOBETA} format. 
\end{xen}

\subsection{\xlang{Directive BUILDZX}{BUILDZX Directive}}\index{BANK}\label{ZXBANK}
\begin{verbatim}
BUILDZX
\end{verbatim}

\begin{xfr}
Cette directive sert a activer la génération de snapshot.
A noter qu'il faut obligatoirement ajouter un second parametre à la directive \texttt{RUN}, afin de spécifier l'adresse de la pile (\texttt{SP})
\end{xfr}

\begin{xen}
Use this directive to enable ZX SNA file generation (Snapshot) 
With ZX SNA, you must specify a second parameter to \texttt{RUN} directive, in order to provide the value of Stack Pointer (\texttt{SP}) 
\end{xen}

\begin{code}
BUILDZX
BANK 0
RUN \#8000,\#6000 ; Entry point=\#8000, SP=\#6000
\end{code}

\subsection{\xlang{Options spécifiques}{Options}}

\begin{xfr}
L'option \texttt{-sx} permet de réaliser un export pour certaines émulateurs ZX, en précisant la banque où le symbole est logé (\texttt{(<bank>:<adresse>)})
\end{xfr}

\begin{xen}
\texttt{-sx} option can be used for exporting symbols for ZX emulators, where the selected bank is output. \texttt{(<bank>:<address>)}
\end{xen}
