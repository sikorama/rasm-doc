\subsubsection{REPEAT}\index{REPEAT}\index{REPEAT\_COUNTER}\index{REND}\index{UNTIL}

\begin{verbatim}
REPEAT <number of repetitions>[,counter] ... REND
REPEAT ... UNTIL <condition>
\end{verbatim}

\begin{xfr}
Répète un bloc d'instructions.
On peut soit fixer un nombre de répétitions, soit utiliser le mode conditionnel avec la directive de fin de bloc \texttt{UNTIL}. Pour compatibilité avec Vasm, il est possible de finir un tel bloc par \texttt{ENDREP} ou \texttt{ENDREPEAT}
\end{xfr}

\begin{xen}
This directive repeats a block of instructions. You may fix a number of repetition or use conditionnal mode with \texttt{UNTIL}. It is also possible to close such a bloc with \texttt{ENDREP} or \texttt{ENDREPEAT}, for Vasm compatibility.
\end{xen}

\begin{code}
cnt=90
repeat
\    defb 64*sin(cnt)
\    cnt=cnt-4
until cnt\textless 0
\end{code}

\begin{xfr}
Dans le cas de répétition non conditionnelle, il est possible de spécifier une variable qui contiendra le numéro d'itération.
Cette variable (ici 'cnt') n'a pas besoin d'éxister, elle sera créée le cas echéant, au début de la boucle \texttt{REPEAT}.
\end{xfr}

\begin{xen}
In the case of a non conditional loop, it is possible to specify a variable which contains the iteraction counter.
There's no need to declare this variable (here 'cnt') prior to the \texttt{REPEAT} block. It will automatically be created.
\end{xen}

\begin{code}
repeat 10,cnt
\ ldi
\ print cnt
rend
\end{code}

% Compteur interne
\begin{xfr}
La variable interne \texttt{REPEAT\_COUNTER} contient le numéro de l'itération courante (en commencant par 1).
\end{xfr}
\begin{xen}
You can get the internal loop counter anytime with internal variable \texttt{REPEAT\_COUNTER}.
\end{xen}

\begin{code}
repeat 10
\ ldi
\ print repeat\_counter
rend
\end{code}


\subsubsection{WHILE, WEND}\index{WHILE}\index{WEND}\index{WHILE\_COUNTER}
\begin{verbatim}
WHILE <condtion> ... wEND
\end{verbatim}

\begin{xfr}
Répete un bloc tant que la condition est évaluée a vrai. A tout moment, il est possible d'utiliser la variable interne \texttt{WHILE\_COUNTER} pour récuperer le compteur de boucle.
\end{xfr}

\begin{xen}
Repeat a block as long as the condition is evaluated as true. You may use the internal variable \texttt{WHILE\_COUNTER} variable to get the loop counter.
\end{xen}

\begin{code}
cpt=10
while cpt\textgreater 0
\ ldi
\ cpt=cpt-1
\ print 'cpt=',cpt,' while\_counter=',while\_counter
wend
\end{code}

\begin{xfr}
La boucle va s'executer 10 fois, avec le résultat suivant sur la sortie standard:
\end{xfr}

\begin{xen}
This code will loop 10 times with the following output:
\end{xen}

\begin{verbatim}
Pre-processing [while.asm]
Assembling
cpt= 9.00 while_counter= 1.00
cpt= 8.00 while_counter= 2.00
cpt= 7.00 while_counter= 3.00
cpt= 6.00 while_counter= 4.00
cpt= 5.00 while_counter= 5.00
cpt= 4.00 while_counter= 6.00
cpt= 3.00 while_counter= 7.00
cpt= 2.00 while_counter= 8.00
cpt= 1.00 while_counter= 9.00
cpt= 0.00 while_counter= 10.00
Write binary file rasmoutput.bin (20 bytes)
\end{verbatim}

