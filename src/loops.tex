\subsubsection{REPEAT}\index{REPEAT}\index{REPEAT\_COUNTER}\index{REND}\index{UNTIL}

\begin{verbatim}
REPEAT <number of repetitions>[,counter[,counter_start[,counter_step]]] ... REND
REPEAT ... UNTIL <condition>
\end{verbatim}

\begin{xfr}
Répète un bloc d'instructions.
On peut soit fixer un nombre de répétitions, soit utiliser le mode conditionnel avec la directive de fin de bloc \texttt{UNTIL}. Pour compatibilité avec Vasm, il est possible de finir un tel bloc par \texttt{ENDREP} ou \texttt{ENDREPEAT} au lieu de \texttt{REND}
\end{xfr}

\begin{xen}
This directive repeats a block of instructions. You may fix a number of repetition or use conditionnal mode with \texttt{UNTIL}. It is also possible to close such a bloc with \texttt{ENDREP} or \texttt{ENDREPEAT}, for Vasm compatibility.
\end{xen}

\begin{code}
cnt=90
repeat
\    defb 64*sin(cnt)
\    cnt=cnt-4
until cnt\textless 0
\end{code}


\begin{xfr}
\paragraph{Répétition non conditionelle}
Dans le cas de répétition non conditionnelle, il est possible de spécifier une variable qui contiendra le numéro d'itération.
Il n'est pas nécessaire de créer cette variable au préalable.
\end{xfr}

\begin{xen}
\paragraph{Non Conditional Repeat}
In the case of a non conditional loop, it is possible to specify a variable which contains the iteraction counter.
There's no need to declare this variable (here \texttt{cnt}) prior to the \texttt{REPEAT} block. It will automatically be created.
\end{xen}

\begin{code}
repeat 10,cnt
\ ldi
\ print cnt
rend
\end{code}

\begin{xfr}
Par défaut le compteur de boucle est initialisé à 1 et est incrément de 1 à chaque boucle.
Il aussi possibe de préciser la valeur de départ avec \texttt{counter\_start}, ainsi que le pas d'incrément à l'aide du paramètre \texttt{counter\_step}
Ces valeurs peuvent aussi bien etre des entiers que des des nombres flottants.
\end{xfr}

\begin{xen}
By default, the loop counter starts from 1 and is increased by 1 at every loop. 
These values can be changed, respectively with \texttt{counter\_start} and \texttt{counter\_step} optional parameters.
These values can be integer, but also floating values.
\end{xen}

\begin{code}
repeat 26,letter,65
\ db letter
rend
\end{code}

\begin{xfr}
Ces valeurs peuvent aussi bien être des entiers que des des nombres flottants.
\end{xfr}

\begin{xen}
These values can either be integer, but also floating values.
\end{xen}

\begin{code}
repeat 180,angle,0,3.14/180
\ db sin(angle)*64+64
rend
\end{code}
    

\paragraph{\xlang{Symbole}{Symbol}REPEAT\_COUNTER}

\begin{verbatim}
REPEAT\_COUNTER [counter_start[,counter_step]]
\end{verbatim}
    
\begin{xfr}
Une alternative à l'utilisation de texttt{counter\_start} et \texttt{counter\_step}, consiste à utiliser  \texttt{REPEAT\_COUNTER}
Ceci permet de ne pas avoir à respécifier ces valeurs pour tous les blocs \texttt{REPEAT} qui suivent.
En l'absence de paramètres, le compteur commencera à 1 et l'incrément sera de 1.
\end{xfr}

\begin{xen}
Another way to specify texttt{counter\_start} and \texttt{counter\_step}  consists in using \texttt{REPEAT\_COUNTER}.
All REPEAT blocks declared after \texttt{REPEAT\_COUNTER} will use these values.
Without parameters, the counter will start at 1, and increment will be set to 1.
\end{xen}

\begin{code}
y=10
repeat 3,x,10
  assert x==y
  y+=5
rend
\end{code}
    
\begin{xfr}
La variable interne \texttt{REPEAT\_COUNTER} contient le numéro de l'itération courante (en commencant par 1).
\end{xfr}
\begin{xen}
You can get the internal loop counter anytime with internal variable \texttt{REPEAT\_COUNTER}.
\end{xen}

\begin{code}
repeat 10
\ ldi
\ print repeat\_counter
rend
\end{code}

    


\subsubsection{WHILE, WEND}\index{WHILE}\index{WEND}\index{WHILE\_COUNTER}
\begin{verbatim}
WHILE <condtion> ... wEND
\end{verbatim}

\begin{xfr}
Répete un bloc tant que la condition est évaluée a vrai. A tout moment, il est possible d'utiliser la variable interne \texttt{WHILE\_COUNTER} pour récuperer le compteur de boucle.
\end{xfr}

\begin{xen}
Repeat a block as long as the condition is evaluated as true. You may use the internal variable \texttt{WHILE\_COUNTER} variable to get the loop counter.
\end{xen}

\begin{code}
cpt=10
while cpt\textgreater 0
\ ldi
\ cpt=cpt-1
\ print 'cpt=',cpt,' while\_counter=',while\_counter
wend
\end{code}

\begin{xfr}
La boucle va s'executer 10 fois, avec le résultat suivant sur la sortie standard:
\end{xfr}

\begin{xen}
This code will loop 10 times with the following output:
\end{xen}

\begin{verbatim}
Pre-processing [while.asm]
Assembling
cpt= 9.00 while_counter= 1.00
cpt= 8.00 while_counter= 2.00
cpt= 7.00 while_counter= 3.00
cpt= 6.00 while_counter= 4.00
cpt= 5.00 while_counter= 5.00
cpt= 4.00 while_counter= 6.00
cpt= 3.00 while_counter= 7.00
cpt= 2.00 while_counter= 8.00
cpt= 1.00 while_counter= 9.00
cpt= 0.00 while_counter= 10.00
Write binary file rasmoutput.bin (20 bytes)
\end{verbatim}

