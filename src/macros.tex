\subsubsection{\xlang{Définition d'une MACRO}{Macros}}\index{MACRO}\label{MACRO}\index{MEND}\index{ENDM}
\begin{verbatim}
MACRO <macro_name> [param1[,param2[,...]]]
...
MEND
\end{verbatim}

\begin{xfr}
Une macro est une facon d'étendre le langage, en définissant un bloc d'instructions, délimité par \texttt{MACRO} et \texttt{MEND} (ou \texttt{ENDM}), et qui pourra être inséré ulterieurement dans le code, par simple utilisation du nom de la macro.
Les macros peuvent prendre des parametres, il est ainsi possible de faire de l'assemblage conditionnel avec les macros : a  chaque appel de macro, le code d'origine est inséré, avec substitution des paramètres. Ce n'est qu'ensuite que le code interprété de façon classique.

Voici un exemple pour une écriture longue distance générique (sauf pour B ou C):

\end{xfr}

\begin{xen}
A macro is a way to extend the language, by defining a block of instructions, delimited by MACRO and \texttt{MEND} (or ENDM) directives, that can later be inserted in your code.
Macros can take parameters, so you can make conditionnal assembling with it: a macro is barely a copy/paste with arguments replacement. Arguments inside the macro are referenced using curly brackets. Here is an example of a long distance indexing, working for any 16 bit register (except B or C):
\end{xen}

\begin{code}
macro LDIXREG register,dep
\ if \{dep\}\textless-128 || \{dep\}\textgreater127
\ \ push BC
\ \ ld BC,\{dep\}
\ \ add IX,BC
\ \ ld (IX+0),\{register\}
\ \ pop BC
\ else
\ \ ld (IX+\{dep\}),\{register\}
\ endif
mend
\end{code}

\paragraph{\xlang{Utilisation d'une macro}{Macro invocation}}

\begin{code}
LDIXREG H,200
LDIXREG L,32
\end{code}

\begin{xfr}
Attention! Rasm ne peut pas savoir si vous voulez déclarer un label quand vous vous trompez dans l'écriture du nom d'une macro sans paramètre.
Pour palier à ce défaut, vous pouvez ajouter un paramètre fictif "\texttt{(void)}"  qui déclenchera une erreur si le nom de macro n'est pas connu.

De plus, si vous souhaitez forcer l'utilisation de cette syntaxe, il est possible de lancer RASM avec l'option \texttt{-void}.
L'utilisation d'une macro sans parametre provoquera une erreur.
\end{xfr}

\begin{xen}
Beware that RASM will understand any misspeled macro as a label declaration!
A recommanded usage for macros without parameters, is to systematicaly use an empty parameter "\texttt{(void)}" .
A misspeled macro call will then trigger an error.
If you want to enforce usage of this syntax, you can use \texttt{-void} option. Using a macro without any parameter will raise an error.

\end{xen}

\begin{code}
MACRO withoutparam
nop
MEND
withoutparam (void) ; secured call of macro
\end{code}

\paragraph{\xlang{Appel de macro avec parametres dynamiques}{Macro calls with static or dynamic args}}

\begin{xfr}
Les parametres passés lors de l'appel d'une macro ne sont pas nécessairement des constantes, ils peuvent etre des expressions:
\end{xfr}

\begin{xen}
Arguments sent to a macro can also be formulas. %either as text replacement, or as formula. %Evaluation that will be evaluated before substitution.
\end{xen}

\begin{code}
MACRO test myarg
\ DEFB \{myarg\}
MEND

;\xlang{Identique à }{Same as } \texttt{defb 1 : defb 2}:
REPEAT 2
\ test repeat\_counter
REND
;\xlang{Identique à }{Same as } \texttt{defb 1 : defb }1:

REPEAT 2
\ test \{eval\}repeat\_counter
REND
\end{code}
