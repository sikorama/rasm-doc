\section{\xlang{Compilation}{Compiling and Embedding}}

\begin{xfr}
%RASM ne nécessite aucune installation, il se résume à un unique fichier binaire éxécutable.
Cette section décrit la procédure pour compiler RASM sur différentes plateformes, le code source C étant disponible. De plus, il est possible d'integrer RASM dans vos propres applications, ce qui sera vu a la prochaine section.
\end{xfr}

\subsection{\xlang{Compiler RASM}{Building RASM}}

\begin{xfr}
Pour commencer, voici les commandes à exécuter pour quelques plateformes courantes.
Pour simplifier la compilation, ce sont des scripts Makefile ou des scripts batch. 
Tous ces scripts font appels à UPX, un outil de compression d'executable, disponible ici: https://upx.github.io
\end{xfr}

\begin{xen}
There is no installation procedure for RASM, as it consists in a single executable file. As C source code is provided with RASM, it can be recompiled. 
Here are the commands for some platforms and compilers - mainly calling scripts or Makefiles 
All these scripts use UPX, a compression tool for executable binaries, available here: https://upx.github.io
\end{xen}

\paragraph{Linux}

\begin{verbatim}
#Makefile invocation
make prod
\end{verbatim}

\paragraph{Windows (Visual Studio)}

\begin{verbatim}
compil.bat
\end{verbatim}

\paragraph{Dos/Windows 32 (Watcom)}

\begin{xfr}
Pour DOSBox, il faudra allouer au minimum 64Mb de RAM.
\end{xfr}
  
\begin{xen}
For DosBox, you need to set RAM to at least 64Mb
\end{xen}

\begin{verbatim}
msdos.bat
\end{verbatim}

\paragraph{MacOS}

\begin{verbatim}
make makefile.MacOS
\end{verbatim}






\subsection{\xlang{Integrer RASM}{Embedding RASM}}

\begin{xfr}
Pour integrer RASM dans votre propre application C/C++, il faut procéder en trois étapes:

\begin{itemize}
\item Compiler RASM sous forme de binaire (obj), avec le symbole \texttt{INTEGRATED\_ASSEMBLY} défini
\item Inclure 'rasm.h' et utiliser l'une des deux fonctions pour compiler du code, ainsi que les structures de données définies dans ce header,si l'on souhaite récuperer des informations sur le processus d'assemblage
\item Compiler votre programme en linkant le binaire produit à l'étape 1.
\end{itemize}

Dans notre premier exemple, nous allons utiliser la fonction \texttt{RasmAssemble}, qui renvoie,en plus du code assemblé,un code d'erreur, 0 si tout s'est bien passé, -1 sinon. La fonction \texttt{RasmAssembleInfos} est identique, à ceci pres qu'elle renvoie une structure de donnée contenant des informations sur le processus d'assemblage: les éventuelles erreurs,et la valeur associé à chacun des symboles.
\end{xfr}

\begin{xen}
There are 3 steps t follow for integrating  RASM into your own C/C++ application:
\begin{itemize}
\item Build RASM as an object binary that can be linked. In order to do this, it must be compiled with  \texttt{INTEGRATED\_ASSEMBLY} symbol.
\item Include 'rasm.h' in our program, and use one of the two functions for assembling our Z80 code.
\item Building our application, without forgetting to link the binary file produced in step 1.
\end{itemize}

As a first example, we'll use \texttt{RasmAssemble} function, which returns an error code (0 if everything went fine, or -1 if an error happened), and also assembled byte code, in an array of unsigned chars.
\texttt{RasmAssembleInfos} is similar, but it also returns additional data, such as a list of errors and the value of all symbols.
\end{xen}


\begin{verbatim}
#include "rasm.h"
#include <stdio.h>
#include <stdlib.h>
#include <string.h>

// Prgram to assemble
const char* prog="org #9000 \n\
    ld b,10 \n\
lp: djnz lp \n\
    ret \n";

int main(void) {
  printf("%s\n", prog);

  unsigned char *buf = NULL;
  int asmsize = 0;
  int res = RasmAssemble(prog, strlen(prog), &buf, &asmsize);

  printf("Result=%d, Generated code size=%d\n",res,asmsize);

  if (res==0)
  {
      for (int i=0; i<asmsize; i++)
      {
        printf("%02X ", buf[i]);
        if ((i&15)==15) printf("\n");
      }
      printf("\n");
    }
  else
    {
      printf("Failure!\n");
    }

  if (buf)
    free(buf);

  return 0;
}
\end{verbatim}

\xlang{Pour compiler ce fichier (embed.cpp) sous linux avec gcc:}{In order to copile rasm and our example (embed.cpp):}

\begin{code}
gcc -D INTEGRATED\_ASSEMBLY rasm\_v0111.c -c -o rasm\_embedded.obj
gcc embed.cpp rasm\_embedded.obj -lm
\end{code}

\xlang{Le résultat de l'éxecution est le suivant:}{When executed, it produces this:}

\begin{code}
\$ ./a.out
org \#9000
    ld b,10
lp: djnz lp
    ret

Result=0, Generated code size=5
06 0A 10 FE C9
\end{code}

\subsubsection{\xlang{Récupération des erreurs et symboles}{Errors and Symbols}}
\begin{xfr}
Dans l'exemple précédent, on a vu que le résultat de la fonction RasmAssemble était -1 en cas d'erreur. Si l'on souhaite d'avantages d'informations, il faudra appeler \texttt{RasmAssembleInfos} qui renvoie une structure contenant la liste des erreurs ainsi que la liste des symboles avec leur valeur.


\end{xfr}
